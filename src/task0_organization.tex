\subsection{Task 0: Organization}\label{sec:organization}

\begin{note}
For this task, please write your code in the file \texttt{organization.yaml}
at the root of the \texttt{workshop/} directory.
\end{note}

The objective of this first task is to create an organization.
All resources created in the future will be part of this organization.
As this is the first task, every step of the process is documented.\\

Once all the steps from \fullref{sec:set} are done, you can declare your organization resource.
Create \texttt{organization.yaml} and write the following configuration block into the file.
\begin{minted}[gobble=4, frame=single ]{yaml}
    apiVersion: organization.taikun.jet.crossplane.io/v1alpha1
    kind: Organization
    metadata:
        name: myorg
    spec:
        forProvider:
            name: <name>
            fullName: <full-name>
            discountRate: 142
        providerConfigRef:
            name: providerconfig-workshop
\end{minted}
Be sure to replace \texttt{<name>} and \texttt{<full-name>} with
names of your choosing.
You can also choose another metadata name instead of \texttt{myorg}.

\begin{tip}
Notice the syntax of the configuration block, as you are creating a resource,
it begins with the keyword \texttt{apiVersion}, followed by its CRD name and the api version.
The type of resource is always lowercase and followed by the name of the provider,
thus \texttt{"organization.taikun.jet.crossplane.io/v1alpha1"}.
Following the resource's \texttt{apiVersion} is a metadata name, it must be unique for this type of resource, and is used
to refer to this specific resource, as you will find out later.
Watch out, this field is not the name of the resource in Taikun.\\

Three arguments are then defined in \texttt{spec} then \texttt{forProvider} fields: \texttt{name}, \texttt{fullName} and \texttt{discountRate}.
On the left side of the colon is the argument's identifier, on the right is its value.
See the \href{https://doc.crds.dev/github.com/itera-io/provider-jet-taikun/organization.taikun.jet.crossplane.io/Organization/v1alpha1}{documentation} of Taikun's organization resource for a full list of arguments, i.e. the resource's \textit{schema}.\\

Metadata names and argument names can contain letters, digits, underscores and hyphens and may not start with a digit. Their length must be lower than 30 characters.
\end{tip}

Now apply your changes.

\begin{shell}
kubectl apply -f organization.yaml
\end{shell}
\begin{tip}
If you have already created resources, \texttt{kubectl apply} will create a new resource
by making a request to Taikun's API.
\end{tip}


To check if your resource is created successfully execute the following command :
\begin{shell}
watch kubectl get managed
\end{shell}

You should get a False state.
\begin{raw}
NAME                                                      READY SYNCED EXTERNAL-NAME AGE
organization.organization.taikun.jet.crossplane.io/myorg        False                15s
\end{raw}
In order to show the errors describe the resource you have just created.
\begin{shell}
kubectl describe organizations.organization.taikun.jet.crossplane.io myorg
\end{shell}

You should have in the \texttt{Event} field at the bottom of the output an error message like the following:
\begin{raw}
Warning  CannotObserveExternalResource  2s (x4 over 7s)  managed/organization.taikun.
jet.crossplane.io/v1alpha1,kind=organization  cannot run refresh: 
refresh failed: expected discount_rate to be in the range (0.000000 - 100.000000),
got 142.000000:
\end{raw}

\begin{tip}
\texttt{kubectl describe} command is used to describe (obviously) the resource you have created. You can use this command in either way:
\begin{shell}
kubetcl describe crd-name my-ressource
kubectl describe -f path-to-my-ressource-file
\end{shell}
To get the name of the crd you can use the \texttt{kubectl get crds} command or the \texttt{kubectl get crds | grep taikun} to see all taikun crds.
\end{tip}

Now fix the discount rate so it is in the range 0-100 and run \texttt{kubectl apply} again.
Normally you should have the following output for \texttt{watch kubectl get managed} command.
\begin{raw}
NAME                                                      READY SYNCED EXTERNAL-NAME AGE
organization.organization.taikun.jet.crossplane.io/myorg  True  True   14236         15s
\end{raw}

\begin{tip}
The \texttt{EXTERNAL-NAME} field is the id of your resource in Taikun.
\end{tip}

You may wish to check the organization was indeed created at
\href{https://app.taikun.cloud/organizations}{app.taikun.cloud} 
(or  \href{https://app.taikun.dev/organizations}{app.taikun.dev} if you are working on the working environment).

\begin{note}
Some resources can take a certain amount of time to be created.
\end{note}