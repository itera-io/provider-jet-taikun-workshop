\subsection{Task 5: Project and User attachment}\label{sec:project}

\begin{note}
For this task, please write all your code in the file \texttt{project.yaml}
at the root of the \texttt{workshop/} directory.
\end{note}

\subsubsection{Project}

Finally, you can declare a project resource.
In order to create the resource you will need to import some flavors and images to 
create a kubernetes cluster or a vm.
In our case we will create a kubernetes cluster so we just need flavors.

\begin{tip}
Please read this \href{https://itera.gitbook.io/taikun/guidelines/creating-a-cluster}{page}.
It explains how to create a cluster with Taikun and which resources are required.
\end{tip}

Now it is your turn to create the project with the resources we created in the previous tasks. 
You can use reference arguments because all the resources needed have been created within your cluster.\\

You can find \href{https://doc.crds.dev/github.com/itera-io/provider-jet-taikun/project.taikun.jet.crossplane.io/Project/v1alpha1}{here} the project resource documentation.

\subsubsection{Project User Attachment}

Now, as you just have written the configuration file of your project, you can assign users to it.
The users you want to attach to the project must be in the same organization. In our case the 
resources must belong to the organization created in \fullref{sec:organization}.\\

See the Project User Attachment documentation \href{https://doc.crds.dev/github.com/itera-io/provider-jet-taikun/projectuserattachment.taikun.jet.crossplane.io/UserAttachment/v1alpha1}{here}, and
after writing the two resources in \texttt{project.yaml}, you can apply your file.

