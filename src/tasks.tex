\section{Documentation}
The provider documentation is available \href{https://doc.crds.dev/github.com/itera-io/provider-jet-taikun}{here}.

\section{Tasks}
The end goal of this workshop is to have an operational Taikun project built solely with Crossplane
configuration files.
By following a step by step process, you will discover how various Taikun
resources are declared and managed using Crossplane.\\

All your work will be done in the \texttt{workshop/} directory. These are its initial contents.
\begin{raw}
./workshop/
|-- providerconfig.yaml
|-- taikun_secret.yaml.tmpl
|-- organization.yaml
|-- kubeprofile.yaml
|-- slack_alerting.yaml
|-- cloudcred.yaml
|-- user_ap.yaml
|-- project.yaml
\end{raw}
\texttt{providerconfig.yaml} contains the Provider configuration,
namely its source address and what credentials to use.
You will not need to edit this file.
\begin{minted}[gobble=4, frame=single ]{yaml}
    apiVersion: taikun.jet.crossplane.io/v1alpha1
    kind: ProviderConfig
    metadata:
        name: providerconfig-workshop
    spec:
        credentials:
            source: Secret
            secretRef:
                name: my-creds
                namespace: crossplane-system
                key: credentials
\end{minted}

During this workshop, each task should be coded in a separate config file.
At the end of the workshop, your directory will be organized as such:
\begin{raw}
./workshop/
|-- providerconfig.yaml
|-- taikun_secret.yaml.tmpl
|-- taikun_secret.yaml
|-- openstack_secret.yaml
|-- organization.yaml
|-- kubeprofile.yaml
|-- slack_alerting.yaml
|-- cloudcred.yaml
|-- user_ap.yaml
|-- project.yaml
\end{raw}

\subsection{Provider and cluster setup}\label{sec:set}

\subsubsection{Prepare the cluster to use the provider}
First thing to do is to install crossplane in your cluster. You can follow \href{https://crossplane.io/docs/v1.9/getting-started/install-configure.html#install-crossplane}{this guide} to install crossplane with Helm.

Once it is done you can install the Taikun crossplane provider with this command at the root of the provider repository:
\begin{shell}
kubectl apply -f examples/install.yaml
\end{shell}

You can now go into the workshop repository.
\begin{shell}
cd path-to-workshop-repo/workshop
\end{shell}

\subsubsection{Authentication}
In order to complete the tasks that follow, you will need to provide Taikun credentials to Crossplane.
You will need a Partner account as some of the tasks, such as creating an organization,
require Partner privileges.\\

\begin{note}
If you want to use the development environment, you must define the \texttt{api\_host} argument under your Taikun password in the secret:
\begin{minted}[gobble=4, frame=single ]{yaml}
    # taikun_secret.yaml.tmpl
    [...]
    "password": "your_password",
    "api_host": "api.taikun.dev"
\end{minted}
if you set the \texttt{api\_host} argument, be sure you have added a coma after the value of the \texttt{password} argument.
\end{note}

Then you can run the following command to create a secret which stores your Taikun credentials to which the providerconfig refers.
Please change \texttt{TAIKUN\_EMAIL} and \texttt{TAIKUN\_PASSWORD} values with your Taikun credentials.
\begin{shell}
TAIKUN_EMAIL="your-email"
TAIKUN_PASSWORD="your-password"
cat taikun_secret.yaml.tmpl | sed -e "s/your_email/${TAIKUN_EMAIL}/g" \
| sed -e "s/your_password/${TAIKUN_PASSWORD}/g" > taikun_secret.yaml
\end{shell}
To find out more about providing sensitive data in Kubernetes, see this \href{https://kubernetes.io/docs/concepts/configuration/secret/}{page}.\\

Finally, in order to connect to Taikun with the Crossplane provider you can now execute the following commands:
\begin{shell}
kubectl apply -f taikun_secret.yaml
kubectl apply -f providerconfig.yaml
\end{shell}

\begin{note}
You will may be use the \texttt{sudo} permission to run \texttt{kubectl} commands.
\end{note}

You can now create Taikun resources !