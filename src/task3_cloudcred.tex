\subsection{Task 3: Cloud Credentials}\label{sec:cloudcred}

\begin{note}
For this task, please write your code in the file \texttt{cloudcred.yaml}
at the root of the \texttt{workshop/} directory.\\
\end{note}

\begin{warn}
You will need OpenStack credentials to complete this task.
\end{warn}

Cloud credentials are needed to create a Taikun project.
In a real work environment,
cloud credentials should not be stored under version control;
it's why we'll use kubernetes secret.\\

Define the OpenStack cloud credential resource in \texttt{cloudcred.yaml}.
 \href{https://github.com/nivraph/provider-jet-taikun/blob/main/docs/cloudcredential.taikun.jet.crossplane.io.md}{Here} is its documentation.

You will need a secret to provide sensitive data like your Openstack password.
For this, create a new file named \texttt{openstack\_secret.yaml} to create a secret 
configuration for Openstack cloud credentials.\\

Here is a secret template :
\begin{minted}[gobble=4, frame=single ]{yaml}
    apiVersion: v1
    kind: Secret
    metadata:
        name: <secret-name>
        namespace: <namespace>
    type: Opaque
    data:
        password: <openstack-password-base64>
\end{minted}

\begin{itemize}
    \item The \texttt{<secret-name>} argument is the name you will refer to during 
    the creation of your Opensatck cloud credential configuration file.
    \item The \texttt{<namespace>} argument is the namespace where your secret will
     be stored.
    \item The \texttt{<openstack-password>} argument is your Openstack password in 
    base 64.
\end{itemize}

\begin{tip}
    To encode your password to base64 you can execute this command 
    \begin{shell}
    echo -n "your-password" | base64
    \end{shell}
    or you can go in a base64 translator like \href{https://www.base64encode.org/}{this one}
\end{tip}


Once you have declared your new resource, apply and move on to the next task.

\begin{warn}
As always, your resources should belong to the organization created in
\fullref{sec:organization}.
\end{warn}

