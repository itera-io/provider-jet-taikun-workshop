\subsection{Task 2: Slack Configuration \& Alerting Profile}\label{sec:slack\_alerting}

\begin{note}
For this task, please write your code in the file \texttt{slack\_alerting.yaml}
at the root of the \texttt{workshop/} directory.
\end{note}

You will now create an alerting profile using a Slack configuration.

\begin{enumerate}
  \item Start by declaring a Slack configuration.
  You can found its documentation in the \texttt{docs/} directory in the 
  \href{https://github.com/itera-io/provider-jet-taikun/}{provider repository}.

Its hook URL should be \texttt{https://hooks.myapp.example/ci}.
It must send \textbf{alert-type notifications only} to the channel \texttt{ci}.
  \item You can now declare the alerting profile.
    \href{https://github.com/itera/provider-jet-taikun/tree/main/docs}{Here} is
     its documentation.\\
    The alerting profile should send notifications \textbf{daily} using the Slack
     configuration declared above.
\end{enumerate}

\begin{warn}
As always, your resources should belong to the organization created in
\fullref{sec:organization}.
\end{warn}

But this time we will use the Reference arguments. Indeed as it is said in the 
\href{https://github.com/itera-io/provider-jet-taikun/tree/main/docs}{documentation}
 of alerting profile resource you have a \texttt{organizationIdRef} arguments and
  the latter allows to refer to an organization by his metadata name. 

For instance, the previous metadata name of the organization you have created is
 \texttt{myorg} (or the name you have chosen). So the \texttt{organizationIdRef}
  argument value will be \texttt{myorg} (or the name you have chosen).

Now you can do the same thing for the argument which refer to the alerting profile
 resource.

\begin{tip}
You can write configuration for multiple resource in one file, you just have to 
separate the resources configurations with \texttt{---}.
\begin{minted}[gobble=4, frame=single ]{yaml}
    apiVersion: organization.taikun.jet.crossplane.io/v1alpha1
    kind: Organization
    metadata:
        name: myorg1
    spec:
        [...]
    ---
    apiVersion: organization.taikun.jet.crossplane.io/v1alpha1
    kind: Organization
        name: myorg2
    spec:
        [...]
\end{minted}
\end{tip}
Once you have declared these two new resources, apply and move on to the next task.
