\subsection{Task 0: Organization}\label{sec:organization}

\begin{note}
For this task, please write your code in the file \texttt{organization.yaml}
at the root of the \texttt{workshop/} directory.
\end{note}

This objective of this first task is to create an organization.
All resources created in the future will be part of this organization.
As this is the first task, every step of the process is documented.

Before you do anything, start by preparing your working directory for other commands.
Be sure that the \fullref{sec:setup} is done before and the command \texttt{make run} 
is blocking and run well.

If all went well, you should see something like the following message.
\begin{raw}
15:41:54 [ .. ] go build linux_amd64
go: downloading ...
go: downloading ...
[...]
15:42:40 [ OK ] go build linux_amd64
15:42:40 [ .. ] Running Crossplane locally out-of-cluster . . .
/home/nivaultr/Documents/Itera/workshop/test-cross-workshop/provider-jet-taikun/_output/
bin/linux_amd64/provider --debug
1.6674001605709026e+09 DEBUG provider-jet-taikun Starting {"sync-period": "1h0m0s"}
I1102 15:42:41.683337  466564 request.go:665] Waited for 1.036556627s due to client-side
throttling, not priority and fairness, request:
GET:https://185.22.97.151:6443/apis/project.taikun.jet.crossplane.io/v1alpha1?timeout=32s
1.6674001619435318e+09 INFO controller-runtime.metrics Metrics server is starting to 
listen {"addr": ":8080"}
1.667400161956083e+09 INFO Starting server {"path": "/metrics", "kind": "metrics",
"addr": "[::]:8080"}
1.6674001619562776e+09 INFO controller.managed/accessprofile.taikun.jet.crossplane.io/
v1alpha1, kind=profile  Starting EventSource {"reconciler group": "accessprofile.
taikun.jet.crossplane.io", "reconciler kind": "Profile", "source": "kind source: 
*v1alpha1.Profile"}
1.6674001619563134e+09 INFO controller.managed/accessprofile.taikun.jet.crossplane.io/
v1alpha1, kind=profile  Starting Controller {"reconciler group": "accessprofile.
taikun.jet.crossplane.io", "reconciler kind": "Profile"}
[...]
\end{raw}

Once all this steps has been done correctly, you can declare your organization resource.
Create \texttt{organization.yaml} and write the following configuration block to it.
\begin{minted}[gobble=4, frame=single, linenos ]{yaml}
    apiVersion: organization.taikun.jet.crossplane.io/v1alpha1
    kind: Organization
    metadata:
        name: myorg
    spec:
        forProvider:
            name: <name>
            fullName: <full-name>
            discountRate: 42
        providerConfigRef:
            name: providerconfig-workshop
\end{minted}
Be sure to replace \texttt{<name>} and \texttt{<full-name>} with
names of your choosing.
You can also choose another metadata name instead of \texttt{myorg}.

\begin{tip}
Notice the syntax of the configuration block, as you are creating a resource,
it begins with the keyword \texttt{apiVersion}, followed by its CRD name and the api
version. The type of resource is always lowercase and followed by the name of the
provider, thus \texttt{"organization.taikun.jet.crossplane.io/v1alpha1"}.
Following the resource's \texttt{apiVersion} is a metadata name, it must be unique 
for this type of resource, and is used to refer to this specific resource, as you 
will find out later. Watch out, this field does not correspond to the name of the 
resource in Taikun.


Three arguments are then defined in \texttt{spec} then \texttt{forProvider} fields: \texttt{name}, \texttt{fullName} and \texttt{discountRate}.
On the left side of the colon is the argument's identifier, on the right is its value.
See the documentation of Taikun's organization resource in the docs directory of the 
\href{https://github.com/itera-io/provider-jet-taikun/}{provider repository} for a 
full list of arguments, i.e. the resource's \textit{schema}.

Metadata names and argument names can contain letters, digits, underscores and hyphens 
and may not start with a digit. Their length must be lower than 30 character.
\end{tip}

Now apply your changes.
\begin{tip}
If you have already created resources, \texttt{kubectl apply} will create a new resource
by making request to Taikun's API.
\end{tip}

\begin{shell}
kubectl apply -f organization.yaml
\end{shell}

To check if your resource is created successfully execute the following command :
\begin{shell}
watch kubectl get managed
\end{shell}

You should get a False state.
\begin{raw}
NAME                                                       READY SYNCED EXTERNAL-NAME AGE
organization.organization.taikun.jet.crossplane.io/myorg   False False                15s
\end{raw}
In order to show the errors describe the resource you have just created.
\begin{shell}
kubectl describe organizations.organization.taikun.jet.crossplane.io myorg
\end{shell}

You should have in the \texttt{Event} field at the bottom of the output an error message
 like the following:
\begin{raw}

TO DO

\end{raw}

\begin{tip}
\texttt{kubectl describe} command is used to describe (obviously) the resource you 
created. You can use it by different way:

\begin{shell}
kubetcl describe crd-name my-ressource
kubectl describe -f path-to-my-ressource-file
\end{shell}
To get the name of the crd you can use the \texttt{kubectl get crds} command or the 
\texttt{kubectl get crds | grep taikun} to see all taikun crds.
\end{tip}

Now fix the discount rate so it is in the range 0-100 and run \texttt{kubectl apply} once
 more. Normally you should have the following output for \texttt{watch kubectl get managed}
  command.

\begin{raw}
NAME                                                      READY SYNCED EXTERNAL-NAME AGE
organization.organization.taikun.jet.crossplane.io/myorg  True  True   14236         15s
\end{raw}

\begin{tip}
The \texttt{EXTERNAL-NAME} field is the id of your resource in Taikun.
\end{tip}

You may wish to check the organization was indeed created at
\href{https://app.taikun.cloud/organizations}{app.taikun.cloud/organizations}.

