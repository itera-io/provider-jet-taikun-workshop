\section{Documentation}
The provider documentation is available in `docs` directory of the \href{https://github.com/itera-io/provider-jet-taikun/}{provider repository}.

\section{Tasks}
The end goal of this workshop is to have an operational Taikun project built solely with Crossplane
configuration files.
By following a step by step process, you will discover how various Taikun
resources are declared and managed using Crossplane.\\

All your work will be done in the \texttt{workshop/} directory. These are its initial contents.
\begin{raw}
./workshop/
|-- providerconfig.yaml
|-- taikun_secret.yaml.tmpl
\end{raw}
\texttt{providerconfig.yaml} contains the Provider configuration,
namely its source address and what credentials to use.
You will not need to edit this file.
\begin{minted}[gobble=4, frame=single, linenos ]{yaml}
    apiVersion: taikun.jet.crossplane.io/v1alpha1
    kind: ProviderConfig
    metadata:
        name: providerconfig-workshop
    spec:
        credentials:
            source: Secret
            secretRef:
                name: my-creds
                namespace: crossplane-system
                key: credentials
\end{minted}

During this workshop, each task should be coded in a separate config file.
At the end of the workshop, your directory will be organized as such:
\begin{raw}
./workshop/
|-- providerconfig.yaml
|-- taikun_secret.yaml.tmpl
|-- taikun_secret.yaml
|-- openstack_secret.yaml.tmpl
|-- openstack_secret.yaml
|-- organization.yaml
|-- kubeprofile.yaml
|-- slack_alerting.yaml
|-- cloudcred.yaml
|-- user_ap.yaml
|-- project.yaml
\end{raw}

\subsection{Provider and cluster setup}\label{sec:auth}
\subsubsection{Authentification}
In order to complete the tasks that follow, you will need to provide Taikun credentials to Crossplane.
You will need a Partner account as some of the tasks, such as creating an organization,
require Partner privileges.\\

Input variables will be explained later in the workshop.
For now, simply edit \texttt{taikun\_secret.yaml.tmpl}
and replace the values of \texttt{email} and \texttt{password}
with your Taikun credentials.
\begin{minted}[gobble=4, frame=single, linenos ]{yaml}
    # taikun_secret.yaml.tmpl
    "email": "your_email",
    "password": "your_password",
\end{minted}

Then you can run the following command to create a secret which stores your Taikun credentials to which the providerconfig refers.
\begin{shell}
TAIKUN_EMAIL="your-email"
TAIKUN_PASSWORD="your-password"
cat taikun_secret.yaml.tmpl | sed -e "s/your_email/${TAIKUN_EMAIL}/g" \
| sed -e "s/your_password/${TAIKUN_PASSWORD}/g" > taikun_secret.yaml
\end{shell}

To find out more about providing sensitive data in Kubernetes, see this \href{https://kubernetes.io/docs/concepts/configuration/secret/}{page}.

\subsubsection{Prepare the cluster to use the provider}
First in the console window where you have clone the provider repository, please write the following command.
\begin{shell}
make submodules
\end{shell}
Then this one to create the crd's:
\begin{shell}
kubectl apply -f package/crds
\end{shell}
Now you can write in the same console window these commands to run against your kubernetes cluster.
\begin{shell}
make run
\end{shell}
If all is working well the previous command is blocking. 
Open a new window and execute the following command to create a new namespace named \texttt{crossplane-system} in your cluster.
\begin{shell}
kubectl create namespace crossplane-system --dry-run=client -o yaml | kubectl apply -f -
\end{shell}
In order to connect to taikun with the Crossplane provider you can now execute the following command:
\begin{shell}
kubectl apply -f examples/providerconfig/
\end{shell}
\begin{warn}
Be sure that you have your kubernetes config file of your cluster in your \texttt{~/.kube/} directory. Otherwise please create this directory and paste your config file like this : \texttt{~/.kube/config}.\\
This place is normally the default one where the kubectl command will look for a kubernetes cluster config. Otherwise you will have may be to set the KUBECONFIG environment variable to this location:
\begin{shell}
export KUBECONFIG="~/.kube/config"
\end{shell}
\end{warn}